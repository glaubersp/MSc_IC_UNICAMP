\chapter{Especificação de uma Biblioteca para a Linguagem HasCASL}
\label{chap:desenvBiblioteca}

\section{Considerações iniciais}

Para capturar todas as características da linguagem \Haskell, a biblioteca deveria utilizar os conceitos de avaliação preguiçosa e funções contínuas, permitindo tipos de dados infinitos.
O uso de tais conceitos exigiria o emprego das construções mais avançadas da linguagem \HasCASL e um profundo conhecimento prévio da linguagem \HOL, para que pudessem ser escritas as provas dos teoremas gerados a serem verificados pelo provador \Isabelle.
Tais exigências inviabilizaram o uso destes conceitos no primeiro contato com a metodologia de especificação algébrica utilizada no projeto.
Decidiu-se, então, iniciar a especificação com tipos de dados com avaliação estrita e, em um refinamento posterior, incluir o uso de tipos com avaliação preguiçosa.

A decisão anterior exigiu que os tipos de dados fossem totalmente reescritos.
Com isso, uma especificação foi criada para representar o tipo booleano.
Desta forma, não foi mais possível utilizar o tipo de dados padrão para \textit{verdadeiro} e \textit{falso}, e nem seus operadores para negação, conjunção e disjunção.
Isto resultou na necessidade de se comparar o resultado de cada sentença com os construtores do novo tipo booleano, tornando a sintaxe visualmente carregada.
Para eliminar estas comparações, seria necessário que o tipo booleano fosse definido a partir do tipo \Verb.Unit., de \CASL, com avaliação preguiçosa.
Dessa forma, tal eliminação foi, também, deixada para um refinamento posterior.

A nomenclatura de tipos e funções da biblioteca \Prelude foi utilizada sempre que possível.
Ao importar um tipo de dado já definido na biblioteca da linguagem \CASL, o mesmo teve seu nome alterado para o respectivo nome na biblioteca \Prelude através do uso da sintaxe apropriada presente em \HasCASL.
Quando novas funções foram especificadas, utilizou-se o nome empregado na biblioteca \Prelude sempre que possível.
No entanto, em alguns casos o nome utilizado na biblioteca \Prelude conflitava com palavras reservadas da linguagem \HasCASL; neste caso, o sufixo \textit{H} foi adicionado ao nome.

As especificações escritas em \HasCASL são orientadas a propriedades, ou seja, as especificações modelam um problema e propriedades são verificadas através da criação e verificação de teoremas.
As especificações são definidas por operadores e predicados, que juntos constituem uma assinatura, e axiomas que governam os elementos da assinatura.
As propriedades da especificação que se deseja verificar são incluídas como teoremas a serem provados utilizando-se um provador de teorema.

As provas escritas em \HOL para \Isabelle frequentemente requerem que axiomas sejam reescritos de forma que possam ser utilizados pelo provador em operações automáticas.
Tais axiomas devem ser reescritos na forma de lemas e devem ser provados da mesma forma que os teoremas.
Alguns dos lemas foram adicionados diretamente às especificações, sendo transformados em teoremas pela ferramenta \Hets; outros, no entanto, foram escritos diretamente em \HOL, por questões de praticidade.
Todos os teoremas, lemas e axiomas foram nomeados para facilitar o uso dos mesmos nas provas geradas para o \Isabelle.

Operadores, tais como $+$, $-$, $*$, $/$, $<$, entre outros, são definidos em \CASL e \Haskell de forma infixa e com tipos \textit{curried}.
Em \Haskell, no entanto, é possível transformar um operador em uma função com tipos \textit{uncurried}, bastando colocar o operador entre parênteses.
A função resultante recebe o nome de seção, em \Haskell.
Para obter o mesmo efeito em \HasCASL, é preciso criar uma $\lambda$-abstração envolvendo o operador, de forma que esta possa ser passada como parâmetro.
Este mecanismo foi utilizado para passar os operadores $<$ e $==$ como parâmetros para funções de segunda ordem que operam sobre listas.

O capítulo apresenta, a seguir, a descrição de cada uma das especificações em uma seção específica.
O código completo das especificações foi agrupado no \citeAppendix{appendix:strictSpec} para que possa ser facilmente consultado.

\section{Bool, a primeira especificação}

Inicialmente, o tipo de dado \Verb.Bool., representando o tipo booleano com construtores para \textit{verdadeiro} e \textit{falso}, foi criado através da importação do tipo de dado \Verb.Boolean. da biblioteca de \CASL, renomeando-se o tipo e as funções para os nomes utilizados na biblioteca \Prelude, como visto a seguir:

\begin{Verbatim}
from Basic/SimpleDatatypes get Boolean
spec Bool = {Boolean with 
         sort Boolean |-> Bool,
         op Not__ |-> not__,
         op __And__ |-> __&&__,
         op __Or__ |-> __||__
        }
then
op otherwise: Bool
. otherwise = True
\end{Verbatim}

Tendo em vista o refinamento para inclusão de tipos com avaliação preguiçosa, decidiu-se recriar o tipo de dado deste o início, uma vez que o tipo importado da biblioteca \CASL possuía apenas tipos com avaliação estrita.
Foram adicionados teoremas que, embora simples, eram necessários para simplificar provas geradas para o \Isabelle.
A especificação final pode ser vista abaixo:

\begin{Verbatim}
spec Bool = %mono
free type Bool ::= True | False 
fun Not__ : Bool -> Bool
fun __&&__ : Bool * Bool -> Bool
fun __||__ : Bool * Bool -> Bool
fun otherwiseH: Bool      
...
end
\end{Verbatim}

\section{\textit{Eq}: especificando a relação de equivalência}
Classes em \HasCASL são similares às de \Haskell.
A declaração de uma classe inclui funções e axiomas sobre variáveis de tipo da classe em questão, representando a interface desta classe.
Uma declaração de instância de tipo obriga um tipo a pertencer a uma classe.
Isto significa que o tipo deve obedecer a todas as declarações de funções da interface da classe, assim como a todos os axiomas.

A especificação \Verb.Eq. (ver \citeAppendix{appendix:strictSpec:eq}), para funções de igualdade e desigualdade, iniciou o uso de classes.
A declaração da classe \Verb.Eq., a qual inclui uma função de igualdade e axiomas para simetria, reflexividade e transitividade desta operação, pode ser vista a seguir.

\begin{Verbatim}
spec Eq = Bool then
class Eq {
var a: Eq
fun __==__ : a * a -> Bool
fun __/=__ : a * a -> Bool
vars x,y,z: a
. x = y => (x == y) = True                   %(EqualTDef)%
. x == y = y == x                            %(EqualSymDef)%
. (x == x) = True                            %(EqualReflex)%
. (x == y) = True /\ (y == z)                
    = True => (x == z) = True                %(EqualTransT)%
. (x /= y) = Not (x == y)                    %(DiffDef)%
...
}                                            
type instance Bool: Eq                       
...
. (False == True) = False                    %(IBE3)%
...
type instance Unit: Eq                       
...
end
\end{Verbatim}

A função de desigualdade é definida usando-se a função de igualdade.
Evitou-se definir a equivalência entre a função de igualdade definida e a função de igualdade padrão de \HasCASL para permitir com que cada tipo de dado definisse o escopo da igualdade.
Foram criados axiomas para verificar o comportamento da função de diferença e teoremas para auxiliar no processo de provas geradas para o \Isabelle.

Os tipos \Verb.Bool. e \Verb.Unit. foram declarados como instâncias da classe \Verb.Eq..
Foi necessário axiomatizar a negação da igualdade entre dois construtores diferentes do tipo \Verb.Bool. porque a igualdade da classe não é equivalente à igualdade padrão de \HasCASL, e é esta última a igualdade utilizada para criar axiomas entre os construtores dos tipos declarados como \Verb.free..
Os demais teoremas puderam ser provados a partir do axioma anterior.
Não foi necessário incluir nenhum axioma na declaração envolvendo o tipo \Verb.Unit. porque este tipo possui apenas um construtor, e as provas decorrem diretamente dos axiomas da classe.

\section{A Especificação \textit{Ord}, para Ordenação}
A próxima especificação criada foi a de relações de ordenação, nomeada \Verb.Ord..
Inicialmente, tentou-se importar a especificação \Verb.Ord. presente na biblioteca de \CASL em \HasCASL/Metatheory/Ord.
No entanto, a especificação existente faz uso de avaliação preguiçosa e, desta forma, não poderia ser importada.
A especificação, então, foi escrita desde o início (ver \citeAppendix{appendix:strictSpec:ord}).

Para criar a especificação \Verb.Ord., o tipo de dado \Verb.Ordering. foi especificado e declarado como instância da classe \Verb.Eq..
Novamente, foi preciso axiomatizar a negação da igualdade entre os três construtores do tipo, dois a dois.

\begin{Verbatim}
spec Ord = Eq and Bool then
free type Ordering ::= LT | EQ | GT
type instance Ordering: Eq
. (LT == EQ) = False  %(IOE04)%
. (LT == GT) = False  %(IOE05)%
. (EQ == GT) = False  %(IOE06)%
\end{Verbatim}

A classe \Verb.Ord. foi criada como subclasse da classe \Verb.Eq., como em \Haskell.
Especificou-se uma relação de ordenação total (\Verb.__<__.), com axiomas para irreflexividade e transitividade, um teorema para assimetria, decorrente dos axiomas anteriores, e um axioma para totalidade, como visto a seguir.

\begin{Verbatim}
class Ord < Eq
{
 var a: Ord
 fun compare: a -> a -> Ordering
 fun __<__ : a * a -> Bool
 fun __>__ : a * a -> Bool
 fun __<=__ : a * a -> Bool
 fun __>=__ : a * a -> Bool
 fun min: a -> a -> a
 fun max: a -> a -> a
 var    x, y, z, w: a
. (x == y) = True => (x < y) = False                      %(LeIrreflexivity)%
. (x < y) = True => y < x = False                   %(LeTAsymmetry)% %implied
. (x < y) = True /\ (y < z) = True                 
=> (x < z) = True                                        %(LeTTransitive)%
. (x < y) = True \/ (y < x) = True                 
\/ (x == y) = True                                            %(LeTTotal)%
\end{Verbatim}

As demais funções de ordenação foram definidas usando-se a relação de ordenação total e a relação de equivalência, quando aplicável, como visto abaixo.
As propriedades de irreflexividade, assimetria, transitividade e totalidade foram definidas como teoremas.

\begin{Verbatim}
 . (x > y) = (y < x)                                                 %(GeDef)%
 . (x <= y) = (x < y) || (x == y)                                   %(LeqDef)%
 . (x >= y) = ((x > y) || (x == y))                                 %(GeqDef)%
\end{Verbatim}

Os axiomas a seguir foram definidos para relacionar as quatro funções de ordenação e a função de equivalência entre sim, para ambos os construtores do tipo de dado \Verb.Bool..
Adicionalmente, foram criados teoremas para verificar que as funções de ordenação se comportam da maneira esperada.

\begin{Verbatim}
%% Relates == and ordering
 . (x == y) = True <=> (x < y) = False /\ (x > y) = False       %(EqTSOrdRel)%
 . (x == y) = False <=> (x < y) = True \/ (x > y) = True        %(EqFSOrdRel)%
 . (x == y) = True <=> (x <= y) = True /\ (x >= y) = True        %(EqTOrdRel)%
 . (x == y) = False <=> (x <= y) = True \/ (x >= y) = True       %(EqFOrdRel)%
 . (x == y) = True /\ (y < z) = True => (x < z) = True       %(EqTOrdTSubstE)%
 . (x == y) = True /\ (y < z) = False => (x < z) = False     %(EqTOrdFSubstE)%
 . (x == y) = True /\ (z < y) = True => (z < x) = True       %(EqTOrdTSubstD)%
 . (x == y) = True /\ (z < y) = False => (z < x) = False     %(EqTOrdFSubstD)%
 . (x < y) = True <=> (x > y) = False /\ (x == y) = False     %(LeTGeFEqFRel)%
 . (x < y) = False <=> (x > y) = True \/ (x == y) = True      %(LeFGeTEqTRel)%
\end{Verbatim}

Finalizando a definição da classe, foram definidas as funções de comparação, máximo e mínimo, como visto a seguir.

\begin{Verbatim}
%% Definitions to compare, max and min using relational operations.
 . (compare x y == LT) = (x < y)                                  %(CmpLTDef)%
 . (compare x y == EQ) = (x == y)                                 %(CmpEQDef)%
 . (compare x y == GT) = (x > y)                                  %(CmpGTDef)%
%% Define min, max                                               
 . (max x y == y) = (x <= y)                                       %(MaxYDef)%
 . (max x y == x) = (y <= x)                                       %(MaxXDef)%
 . (min x y == x) = (x <= y)                                       %(MinXDef)%
 . (min x y == y) = (y <= x)                                       %(MinYDef)%
 . (max x y == y) = (max y x == y)                         %(MaxSym)% %implied
 . (min x y == y) = (min y x == y)                         %(MinSym)% %implied
\end{Verbatim}

As declarações de instância dos tipos \Verb.Ordering. e \Verb.Bool., para a classe \Verb.Ord., seguiram o mesmo padrão das demais: definiu-se o funcionamento da função de ordenação total sobre os construtores de cada tipo; as demais funções foram escritas como teoremas.
O tipo \Verb.Unit., por possuir apenas um construtor, teve todas as funções declaradas como teoremas porque as mesmas podem ser provadas diretamente dos axiomas.

\section{Especificações \textit{Maybe}, \textit{Either}, \textit{MaybeMonad} e \textit{EitherFunctor}}
Os tipos de dado \Verb.Maybe a. e \Verb.Either a b., onde \Verb.a. e \Verb.b. são variáveis de tipo, foram desenvolvidos em duas fases.

No primeiro passo, o tipo de dado \textit{Maybe} (ver \citeAppendix{appendix:strictSpec:maybe}) foI especificado, acompanhado de uma função de mapeamento e declarações de instância para as classes \Verb.Eq. e \Verb.Ord..

\begin{Verbatim}
spec Maybe = Eq and Ord then
var a,b,c : Type;
    e : Eq;
    o : Ord;
free type Maybe a ::= Just a | Nothing
var x : a;
    y : b;
    ma : Maybe a;
    f : a -> b
...
type instance Maybe e: Eq
var x,y : e; 
. (Just x == Just y) = True <=> (x == y) = True        %(IME01)%
...
. Just x == Nothing = False                            %(IME03)%
type instance Maybe o: Ord
var x,y : o;
. (Nothing < Just x) = True                            %(IMO01)%
. (Just x < Just y) = (x < y)                          %(IMO02)%
...
end
\end{Verbatim}



As declarações de classes foram feitas através da aplicação das funções das classes sobre elementos de um mesmo construtor, para os dois construtores existentes em cada tipo, e em seguida, entre dois elementos de construtores diferentes.
As aplicações  da função \Verb.__==__. entre construtores com variáveis de tipo e da função \Verb.__<__. foram inseridas como axiomas.
A igualdade entre construtores sem variáveis de tipo e as demais funções de ordenação foram definidas como teoremas porque seu comportamento pode ser deduzido a partir dos axiomas anteriores.

Tratamento parecido foi utilizado na especificação do tipo de dado \textit{Either} (ver \citeAppendix{appendix:strictSpec:either}).

\begin{Verbatim}
spec Either = Eq and Ord then
var a, b, c : Type; e, ee : Eq; o, oo : Ord;
free type Either a b ::= Left a | Right b     
...
type instance Either e ee: Eq
var x,y : e; z,w : ee;
. ((Left x : Either e ee) == 
   (Left y : Either e ee)) = (x == y)                    %(IEE01)%
. ((Right z : Either e ee) ==
   (Right w : Either e ee)) = (z == w)                   %(IEE02)%
. ((Left x : Either e ee) ==
   (Right z : Either e ee)) = False                      %(IEE03)%
type instance Either o oo: Ord
var x,y : o; z,w : oo;
. ((Left x : Either o oo) < (Right z : Either o oo))
     = True                                              %(IEO01)%
. ((Left x : Either o oo) < (Left y : Either o oo))
     = (x < y)                                           %(IEO02)%
. ((Right z : Either o oo) < (Right w : Either o oo))
     = (z < w)                                           %(IEO03)%
...
end
\end{Verbatim}

Na segunda fase, as especificações foram estendidas para incluir declarações para classes envolvendo functores e mônadas.
O tipo \Verb.Maybe a. (ver \citeAppendix{appendix:strictSpec:maybeMonad}) foi declarado como instância das classes \Verb.Functor. e \Verb.Monad..

\begin{Verbatim}
spec MaybeMonad = Maybe and Monad then
var a,b,c : Type;
         e : Eq;
         o : Ord;
type instance Maybe: Functor
vars  x: Maybe a;
      f: a -> b;
      g: b -> c
. map (\ y: a .! y) x = x                            %(IMF01)% %implied
. map (\ y: a .! g (f y)) x = map g (map f x)        %(IMF02)% %implied
type instance Maybe: Monad
vars  x, y: a;
      p: Maybe a;
      q: a ->? Maybe b;
      r: b ->? Maybe c;
      f: a ->? b
. def q x => ret x >>= q = q x                       %(IMM01)% %implied
. p >>= (\ x: a . ret (f x) >>= r)
     = p >>= \ x: a . r (f x)                        %(IMM02)% %implied
. p >>= ret = p                                      %(IMM03)% %implied
. (p >>= q) >>= r = p >>= \ x: a . q x >>= r         %(IMM04)% %implied
. (ret x : Maybe a) = ret y => x = y                 %(IMM05)% %implied
var x : Maybe a;
    f : a -> b;
. map f x = x >>= (\ y:a . ret (f y))                 %(T01)% %implied
end
\end{Verbatim}

Já o tipo \Verb.Either a b. (ver \citeAppendix{appendix:strictSpec:eitherFunctor}) foi declarado como instância da classe \Verb.Functor..

\begin{Verbatim}
spec EitherFunctor = Either and Functor then
var a, b, c : Type;
    e, ee : Eq;
    o, oo : Ord;
type instance Either a: Functor
vars x: Either c a;  
     f: a -> b;
     g: b -> c
. map (\ y: a .! y) x = x                          %(IEF01)% %implied
. map (\ y: a .! g (f y)) x = map g (map f x)      %(IEF02)% %implied
end
\end{Verbatim}

Esta separação realizada foi necessária devido à falta de suporte, por parte da ferramenta \Hets, para tradução de construtores de classe de ordens superiores de \HasCASL para \HOL.
Ela também permitiu que as provas que não envolviam funtores e mônadas pudessem ser escritas, já que a inclusão das declarações de instância para classes com construtores de segunda ordem impediria que os demais axiomas pudessem ser provados.

\section{Especificando funções através das especificações \textit{Composition} e \textit{Function}}
Antes de especificar funções da biblioteca \Prelude, fez-se necessário escolher entre importar ou recriar a especificação para composição de funções.
Tendo em vista, novamente, que mudanças seriam necessárias quando a especificação fosse refinada para uso de avaliação preguiçosa, a mesma foi reescrita.
Trocou-se o uso da $\lambda$-abstração pela aplicação direta da função porque a aplicação do axioma em questão nas provas geradas para o \Isabelle torna-se mais simples.
Em seguida, foram especificadas funções básicas como a função identidade, funções para operar sobre tuplas e funções para transformar funções \textit{curried} em funções \textit{uncurried} e vice-versa, como visto a seguir.

\begin{Verbatim}
spec Composition =
vars a,b,c : Type
fun __o__ : (b -> c) * (a -> b) -> (a -> c);
vars a,b,c : Type; y:a; f : b -> c; g : a -> b
     . ((f o g) y) = f (g y)         %(Comp1)%
end

spec Function = Composition then
var a,b,c: Type; x: a; y: b; f: a -> b -> c; g: (a * b) -> c
fun id: a -> a
fun flip: (a -> b -> c) -> b -> a -> c
fun fst: (a * b) -> a
fun snd: (a * b) -> b
fun curry: ((a * b) -> c) -> a -> b -> c
fun uncurry: (a -> b -> c) -> (a * b) -> c
. id x = x                                         %(IdDef)%
. flip f y x = f x y                             %(FlipDef)%
. fst (x, y) = x                                  %(FstDef)%
. snd (x, y) = y                                  %(SndDef)%
. curry g x y = g (x, y)                        %(CurryDef)%
. uncurry f (x,y) = f x y                     %(UncurryDef)%
end
\end{Verbatim}

\section{Utilizando especificações numéricas}
Quando do início deste trabalho, não era possível a tradução de um código escrito em \HasCASL para seu equivalente escrito em \HOL quando o primeiro importasse especificações que utilizassem subtipos, como é o caso das especificações numéricas existentes na biblioteca de \CASL.
Reescrever todas as especificações estava fora do escopo do trabalho e geraria um retrabalho exaustivo e desnecessário.
A solução inicial foi remover o uso de subtipos e criar funções de coerção de tipos.
Posteriormente, como a ferramenta \Hets passou a tratar corretamente o uso de subtipos, foi possível abandonar tal estratégia em prol da importação direta das bibliotecas preexistentes em \CASL.

As especificações numéricas escritas em \CASL, embora completas, geram especificações equivalentes em \HOL que não conseguem ser manipuladas pelo provador \Isabelle nos processos de reescrita.
A manipulação destas especificações dependeria da criação de diversos lemas e teoremas auxiliares para que o provador \Isabelle pudesse usar os axiomas destas especificações em processos de reescrita.
Uma outra abordagem seria a crianção de isomorfismos mapeando os tipos e funções destas especificações para os respectivos tipos e funções dos tipos primitivos da linguagem \HOL.
Esta segunda abordagem, além de evitar a duplicação de tipos de dados com fins idênticos, ainda facilitaria as provas envolvendo tipos numéricos, uma vez que \Isabelle possui vários métodos automáticos para lidar com estes tipos.
Embora já exista um isomorfismo no repositório de códigos de \CASL, não foi possível utilizá-lo para as provas por ter sido escrito para uma versão anterior da ferramenta \HOL.
Uma nova versão encontra-se em desenvolvimento.

O uso de especificações numéricas, devido ao exposto anteriormente, aumenta ainda mais a complexidade das provas escritas em \HOL.
Como o projeto prioriza as especificações ao invés das provas, e como o desenvolvimento de provas envolvendo números, dada a complexidade das provas, consumia muito tempo, optou-se por não finalizá-las, concentrando-se os esforços em aumentar a quantidade de especificações escritas.

As especificações numéricas de \CASL foram importadas e agrupadas para que classes numéricas fossem criadas tal qual existem na biblioteca \Prelude (ver \citeAppendix{appendix:strictSpec:numericClasses}).
O primeiro passo consistiu em declarar os tipos numéricos como instâncias das classes \Verb.Eq. e \Verb.Ord., como é visto a seguir.

\begin{Verbatim}
from Basic/Numbers get Nat, Int, Rat

spec NumericClasses = Ord and Nat and Int and Rat then
type instance Pos: Eq
type instance Pos: Ord
type instance Nat: Eq
type instance Nat: Ord
type instance Int: Eq
type instance Int: Ord
type instance Rat: Eq
type instance Rat: Ord
\end{Verbatim}

Criou-se uma classe \Verb.Num., subclasse de \Verb.Eq., com as operações esperadas para tipos numéricos.
Um teorema envolvendo o valor absoluto e o sinal de elementos de tipos numéricos foi definido logo após a classe.

\begin{Verbatim}
class Num < Eq {
 vars a: Num;
 x,y : a
 fun __+__: a * a -> a
 fun __*__: a * a -> a
 fun __-__: a * a -> a
 fun negate: a -> a
 fun abs: a -> a
 fun signum: a -> a
 fun fromInteger: Int -> a
}
vars a: Num;
     x,y : a
. (abs x) * (signum x) = x                            %(AbsSignumLaw)% %implied
\end{Verbatim}

Em seguida, para cada tipo numérico, efetuou-se a declaração de instância da classe \Verb.Num., mapeando-se as operações da classe para as operações definidas nas especificações importadas, fazendo conversões de tipo, quando necessário, através da função de coerção de tipos \Verb.__as__., como visto no caso do tipo \Verb.Nat., a seguir.
\begin{Verbatim}
type instance Nat: Num                                     
vars a: Num;                                               
     x,y: Nat;                                             
     z: Int                                                
. x + y = (__+__: Nat * Nat -> Nat) (x,y)                   %(INN01)%
. x * y = (__*__: Nat * Nat -> Nat) (x,y)                   %(INN02)%
. x - y = (__-!__: Nat * Nat -> Nat) (x,y)                  %(INN03)%
. negate x = 0 -! x                                         %(INN04)%
. (fun abs: a -> a) x = x                                   %(INN05)%
. signum x = 1                                              %(INN06)%
. fromInteger z = z as Nat                                  %(INN07)%
\end{Verbatim}

As classes \Verb.Integral., vista a seguir, e a classe \Verb.Fractional. também foram definidas, embora a primeira ainda não seja subclasse das classes \Verb.Real. e \Verb.Enum., que não foram incluídas na especificação pela complexidade envolvida na manipulação destas e do tipo inteiro de máquina, limitado pelo tamanho da palavra do processador.

\begin{Verbatim}                                                 
class Integral < Num
{
vars a: Integral;
fun __quot__, __rem__, __div__, __mod__: a * a -> a
fun quotRem, divMod: a -> a -> (a * a)
fun toInteger: a -> Int
}

type instance Nat: Integral
type instance Int: Integral
type instance Rat: Integral

vars a: Integral;
     x,y,z,w,r,s: a;
. (z,w) = quotRem x y => x quot y = z                       %(IRI01)%
. (z,w) = quotRem x y => x rem y = w                        %(IRI02)%
. (z,w) = divMod x y => x div y = z                         %(IRI03)%
. (z,w) = divMod x y => x mod y = w                         %(IRI04)%
. signum w = negate (signum y) /\ (z,w) = quotRem x y   
    => divMod x y =                                     
        (z - (fromInteger (toInteger (1:Nat))) , w + s)     %(IRI05)%
. not (signum w = negate (signum y))                    
    /\ (z,w) = quotRem x y                              
        => divMod x y = (z, w)                              %(IRI06)%
\end{Verbatim}

\section{Especificando listas e operações associadas}
A especificação de listas foi a primeira a necessitar de tipos numéricos.
Com base na discussão apresentada anteriormente, decidiu-se dividir a especificação em duas outras, separando-se o tipo de dado e as funções que não necessitavam de tipos numéricos das funções que envolviam estes tipos.
Assim, os teoremas da primeira especificação poderiam ser provados como nas especificações anteriores, e os que envolviam números poderiam não ser provados, provocando um comprometimento mínimo da especificação quanto à sua validação.

A primeira especificação (ver \citeAppendix{appendix:strictSpec:listNoNumbers}) foi dividida em seis partes, a fim de agrupar funções relacionadas da mesma forma como é feito na biblioteca \Prelude.
Primeiramente, definiu-se o tipo de dado \Verb.free type List a., dependente do tipo \Verb.a., com os construtores \Verb.Nil. e \Verb.Cons a (List a)..

\begin{Verbatim}
spec ListNoNumbers = Function and Ord then 
var a : Type 
free type List a ::= Nil | Cons a (List a) 
\end{Verbatim}

Em seguida, definiram-se algumas funções básicas, onde pela primeira vez foi necessário o uso de funções parciais, uma vez que as funções \Verb.head. e \Verb.tail. não são definidas para listas vazias.

\begin{Verbatim}
var a,b : Type 
fun head : List a ->? a; 
fun tail : List a ->? List a; 
fun foldr : (a -> b -> b) -> b -> List a -> b; 
fun foldl : (a -> b -> a) -> a -> List b -> a; 
fun map : (a -> b) -> List a -> List b; 
fun filter : (a -> Bool) -> List a -> List a; 
fun __++__ : List a * List a -> List a; 
fun zip : List a -> List b -> List (a * b); 
fun unzip : List (a * b) -> (List a * List b) 
\end{Verbatim}

Na segunda parte da especificação estão as declarações de instância para as classes \Verb.Eq. e \Verb.Ord..
As funções foram definidas de forma análoga às declarações de instância das especificações anteriormente descritas.

\begin{Verbatim}
var a : Eq; x,y: a; xs, ys: List a
type instance List a: Eq
. ((Cons x xs) == (Cons y ys)) = ((x == y) && (xs == ys))         %(ILE02)%
var b : Ord; z,w: b; zs, ws: List b
type instance List b: Ord
. (z < w) = True => ((Cons z zs) < (Cons w ws)) = True            %(ILO05)%
. (z == w) = True => ((Cons z zs) < (Cons w ws)) = (zs < ws)      %(ILO06)%
. (z < w) = False /\ (z == w) = False                             
     => ((Cons z zs) < (Cons w ws)) = False                       %(ILO07)%
\end{Verbatim}

A terceira parte concentra quatro teoremas que relacionam entre si algumas funções da primeira parte da especificação.
Embora estes teoremas definam como as funções interagem, eles não devem ser axiomas porque são consequência da definição das funções.
O uso da diretiva \Verb.%implies. indica que todas as equações definidas nesta subparte são consideradas teoremas.

\begin{Verbatim}
. foldl i e (ys ++ ts) 
	= foldl i (foldl i e ys) ts       %(FoldlDecomp)% 
. map f (xs ++ zs) 
	= (map f xs) ++ (map f zs)        %(MapDecomp)% 
. map (g o f) xs = map g (map f xs)   %(MapFunctor)% 
. filter p (map f xs) 
	= map f (filter (p o f) xs)       %(FilterProm)%
\end{Verbatim}

% TODO As funções init, last, null e reverse poderiam ser movidas para a primeira seção. Verificar isto em relação à especificação List de CASL.
A quarta seção inclui algumas outras funções básicas, que complementam as funções da primeira seção, além de funções que mapeiam outras funções sobre elementos de uma lista.
Novamente, algumas funções foram definidas como parciais por não serem definidas para listas vazias.
\begin{Verbatim}
fun init: List a ->? List a; 
fun last: List a ->? a; 
fun null: List a -> Bool; 
fun reverse: List a -> List a; 
fun foldr1: (a -> a -> a) -> List a ->? a; 
fun foldl1: (a -> a -> a) -> List a ->? a; 
fun scanl: (a -> b -> a) -> a -> List b -> List a 
fun scanl1: (a -> a -> a) -> List a -> List a 
fun scanr: (a -> b -> b) -> b -> List a -> List b 
fun scanr1: (a -> a -> a) -> List a -> List a 
\end{Verbatim}

A quinta parte apresenta funções lógicas sobre listas e funções responsáveis pela criação de sub-listas, segundo um dado predicado.
\begin{Verbatim}
fun andL : List Bool -> Bool; 
fun orL : List Bool -> Bool; 
fun any : (a -> Bool) -> List a -> Bool; 
fun all : (a -> Bool) -> List a -> Bool; 
fun concatMap : (a -> List b) -> List a -> List b; 
fun concat : List (List a) -> List a; 
fun maximum : List d ->? d; 
fun minimum : List d ->? d; 
fun takeWhile : (a -> Bool) -> List a -> List a 
fun dropWhile : (a -> Bool) -> List a -> List a 
fun span : (a -> Bool) -> List a -> (List a * List a) 
fun break : (a -> Bool) -> List a -> (List a * List a) 
\end{Verbatim}

A última seção da especificação apresenta funções sobre listas que não estão definidas na biblioteca \Prelude, mas estão presentes em outras bibliotecas de \Haskell e que foram necessárias para definir algumas especificações presentes neste trabalho.
\begin{Verbatim}
fun insert: d -> List d -> List d
fun delete: e -> List e -> List e
fun select: (a -> Bool) -> a -> (List a * List a) -> (List a * List a)
fun partition: (a -> Bool) -> List a -> (List a * List a)
\end{Verbatim}

\section{Agrupando listas e funções com tipos numéricos}
A especificação de listas envolvendo tipos numéricos (ver \citeAppendix{appendix:strictSpec:listWithNumbers}), como pode ser vista abaixo, inclui funções para tamanho de listas, criação de sub-listas de tamanho definido, divisão de lista em posição definida e soma e produto de todos os elementos de uma lista.
Duas funções auxiliares, para soma e produto, foram criadas, de forma a facilitar a legibilidade da especificação.
Como não deveriam ser utilizadas fora do escopo desta especificação, as funções foram tornadas invisíveis através da construção \Verb.hide sum', product'..
No entanto, quando a ferramenta \Hets é utilizada para traduzir esta especificação, a mesma sugere que seja utilizada uma versão sem funções ocultas.
Apesar do aviso, as especificações que importam a especificação com elementos ocultos são traduzidas normalmente pela ferramenta \Hets.

\begin{Verbatim}
spec ListWithNumbers = ListNoNumbers and NumericClasses then {
vars a,b: Type;
     c,d: Num;
     x,y : a;
     xs,ys : List a;
     n,nx : Int;
     z,w: Int;
     zs,ws: List Int
fun length: List a -> Int;
fun take: Int -> List a -> List a
fun drop: Int -> List a -> List a
fun splitAt: Int -> List a -> (List a * List a)
fun sum: List c -> c
fun sum': List c -> c -> c
fun product: List c -> c
fun product': List c -> c -> c
...
} hide sum', product'
end
\end{Verbatim}

\section{Adicionando funções numéricas}
Uma abordagem semelhante à empregada para listas foi utilizada para incluir funções que utilizam tipos numéricos (ver \citeAppendix{appendix:strictSpec:numericFunctions}).
Funções de paridade, exponenciação, cálculo de mínimo múltiplo comum e de máximo divisor comum foram definidas estendendo-se a especificação de funções.
Novamente, foram usadas funções temporárias, escondidas das demais especificações, como visto a seguir.

\begin{Verbatim}
spec NumericFunctions = Function and NumericClasses then {
var a: Num;
    b: Integral;
    c: Fractional
fun subtract: a -> a -> a
fun even: b -> Bool
fun odd: b -> Bool
fun gcd: b -> b ->? b
fun lcm: b -> b -> b
fun gcd': b -> b -> b
fun __^__: a * b -> a
fun f: a -> b -> a
fun g: a -> b -> a -> a
...
} hide f,g
end
\end{Verbatim}

\section{Suporte a caracteres e cadeias de caracteres}
Para suportar caracteres e cadeias de caracteres, importou-se a especificação existente na biblioteca de \CASL e adicionaram-se declarações de instância de classes para as classes \Verb.Eq. e \Verb.Ord..
A especificação \Verb.Char., de \CASL, foi renomeada ao ser importada para evitar conflitos de nomes.
Esta especificação mapeia os códigos hexadecimais que representam caracteres no sistema \textit{Unicode} para um elemento identificado por um número natural.
Em seguida, os caracteres propriamente ditos são relacionados com os seus respectivos códigos hexadecimais.
As declarações de classe definem o funcionamento das funções \Verb.__==__. e \Verb.__<__. sobre o construtor do tipo de dado \Verb.Char..
As demais funções, que decorrem das duas definições anteriores,foram marcadas como teoremas, como se vê abaixo.

\begin{Verbatim}
from Basic/CharactersAndStrings get Char |-> IChar

spec Char = IChar and Ord and NumericClasses then
vars x, y: Char
type instance Char: Eq                             
. (ord(x) == ord(y)) = (x == y)                                   %(ICE01)%
. Not(ord(x) == ord(y)) = (x /= y)                       %(ICE02)% %implied
type instance Char: Ord                                  
. (ord(x) < ord(y)) = (x < y)                                     %(ICO04)%
. (ord(x) <= ord(y)) = (x <= y)                          %(ICO05)% %implied
. (ord(x) > ord(y)) = (x > y)                            %(ICO06)% %implied
. (ord(x) >= ord(y)) = (x >= y)                          %(ICO07)% %implied
. (compare x y == EQ) = (ord(x) == ord(y))               %(ICO01)% %implied
. (compare x y == LT) = (ord(x) < ord(y))                %(ICO02)% %implied
. (compare x y == GT) = (ord(x) > ord(y))                %(ICO03)% %implied
. (ord(x) <= ord(y)) = (max x y == y)                    %(ICO08)% %implied
. (ord(y) <= ord(x)) = (max x y == x)                    %(ICO09)% %implied
. (ord(x) <= ord(y)) = (min x y == x)                    %(ICO10)% %implied
. (ord(y) <= ord(x)) = (min x y == y)                    %(ICO11)% %implied
end
\end{Verbatim}

A especificação de cadeias de caracteres define o tipo de dado \Verb.String. como um apelido para o tipo \Verb.List Char. através do operador \Verb.:=..
Com esta definição, não foi criado um novo tipo de dado, mas um outro nome, mais simples, para se referenciar o tipo composto.
Dessa forma, as operações definidas para o tipo \Verb.List. são válidas para o tipo \Verb.String. e nenhuma definição adicional é necessária para que o tipo esteja bem definido.
Alguns teoremas simples foram criados para verificar se o comportamento deste tipo era o esperado.

\begin{Verbatim}
spec String = %mono
     ListNoNumbers and Char then
type String := List Char
vars a,b: String; x,y,z: Char; xs, ys: String
. x == y = True =>
 	((Cons x xs) == (Cons y xs)) = True                   %(StringT1)% %implied
. xs /= ys = True => 
	((Cons x ys) == (Cons y xs)) = False                  %(StringT2)% %implied
. (a /= b) = True =>  (a == b) = False                    %(StringT3)% %implied
. (x < y) = True => 
	((Cons x xs) < (Cons y xs)) = True                    %(StringT4)% %implied
. (x < y) = True /\ (y < z) = True => ((Cons x (Cons z Nil)) 
         < (Cons x (Cons y Nil))) = False                 %(StringT5)% %implied
end
\end{Verbatim}

\section{Definindo listas de tipos monádicos}
Listas de tipos monádicos, muito utilizadas no sequenciamento de ações de entrada e saída em programas \Haskell, foram especificadas a partir das listas sem tipos numéricos.
Funções para sequências de ações e mapeamento de funções monádicas sobre listas foram definidas tal qual na biblioteca \Prelude, como visto abaixo.

\begin{Verbatim}
spec MonadicList = Monad and ListNoNumbers then
vars a,b: Type;
     m: Monad;
     f: a -> m b;
     ms: List (m a);
     k: m a -> m (List a) -> m (List a);
     n: m a;
     nn: m (List a);
     x: a;
     xs: List a;
fun sequence: List (m a) -> m (List a)
fun sequenceUnit: List (m a) -> m Unit
fun mapM: (a -> m b) -> List a -> m (List b)
fun mapMUnit: (a -> m b) -> List a -> m (List Unit)
. sequence ms = let
  k n nn = n >>= \ x:a. (nn >>= \ xs: List a . (ret (Cons x xs))) in
    foldr k (ret (Nil: List a)) ms                      %(SequenceListDef)%
end
\end{Verbatim}

\section{Exemplificando o uso da biblioteca desenvolvida}
A fim de exemplificar o uso da biblioteca desenvolvida, foram criadas duas especificações envolvendo algoritmos de ordenação.
Foram escolhidas funções de ordenação porque envolveriam listas e poderiam não envolver números.
Desta forma, poder-se-ia tentar provar os teoremas, completando os exemplos.

A primeira especificação (ver \citeAppendix{appendix:strictSpec:sortingPrograms}), mais simples, utilizou os algoritmos \textit{Insertion Sort} e \textit{Quick Sort}.
As funções de ordenação foram definidas através de funções da biblioteca criada e de $\lambda$-abstrações totais, usadas como parâmetros para as aplicações da função \Verb.filter..
Para verificar a correção da especificação, foram criados quatro teoremas que aplicam as funções de ordenação, como visto abaixo.

\begin{Verbatim}
spec ExamplePrograms = ListNoNumbers then
var a: Ord;
    x,y: a;
    xs,ys: List a
fun quickSort: List a -> List a
fun insertionSort: List a -> List a
. quickSort (Nil: List a) = Nil                                %(QuickSortNil)%
. quickSort (Cons x xs)                                   
    = ((quickSort (filter (\ y:a .! y < x) xs))           
      ++ (Cons x Nil))                                    
        ++ (quickSort (filter (\ y:a .! y >= x) xs))          %(QuickSortCons)%
. insertionSort (Nil: List a) = Nil                        %(InsertionSortNil)%
. insertionSort (Cons x xs)                               
    = insert x (insertionSort xs)                     %(InsertionSortConsCons)%
then %implies
var a: Ord;
    x,y: a;
    xs,ys: List a
. andL (Cons True (Cons True (Cons True Nil))) = True        %(Program01)%
. quickSort (Cons True (Cons False (Nil: List Bool)))       
     = Cons False (Cons True Nil)                            %(Program02)%
. insertionSort (Cons True (Cons False (Nil: List Bool)))   
     = Cons False (Cons True Nil)                            %(Program03)%
. insertionSort xs = quickSort xs                            %(Program04)%
end
\end{Verbatim}

Na segunda especificação (ver \citeAppendix{appendix:strictSpec:sortingPrograms}), um novo tipo de dado foi criado (\Verb.Split a b.) para ser usado como representação interna nas funções de ordenação.
A ideia utilizada na especificação foi dividir uma lista e, em seguida, unir as suas partes de acordo com o algoritmo escolhido.

\begin{Verbatim}
spec SortingPrograms = ListWithNumbers then
var a,b : Ord;
free type Split a b ::= Split b (List (List a))
\end{Verbatim}

Em seguida, definiu-se uma função de ordenação genérica, chamada \Verb.GenSort., que aplica as funções de divisão e união sobre uma lista.

\begin{Verbatim}
fun genSort:(List a -> Split a b)-> (Split a b -> List a)-> List a -> List a
...
. xs = (Cons x (Cons y ys)) /\ split xs = Split r xxs 
     => genSort split join xs
          = join (Split r (map (genSort split join) xxs))         %(GenSortT1)%
. xs = (Cons x (Cons y Nil)) /\ split xs = Split r xxs 
     => genSort split join xs 
          = join (Split r (map (genSort split join) xxs))         %(GenSortT2)%
. xs = (Cons x Nil) \/ xs = Nil
     => genSort split join xs = xs                                 %(GenSortF)%
. splitInsertionSort (Cons x xs) 
\end{Verbatim}

O algoritmo \Verb.Insertion Sort. foi implementado com o auxílio de uma função responsável por inserir elementos em uma lista através da função \Verb.insert..

\begin{Verbatim}
fun splitInsertionSort: List b -> Split b b
fun joinInsertionSort: Split a a -> List a
fun insertionSort: List a -> List a
...
     = Split x (Cons xs (Nil: List (List a)))           %(SplitInsertionSort)%
. joinInsertionSort (Split x (Cons xs (Nil: List (List a)))) 
     = insert x xs                                        %(JoinInsertionSort)%
. insertionSort xs 
     = genSort splitInsertionSort joinInsertionSort xs    %(InsertionSort)%
\end{Verbatim}

O algoritmo \Verb.Quick Sort. utiliza uma função de divisão que particiona a lista em duas novas listas com o uso da função \Verb.__<__., passada como parâmetro através de uma $\lambda$-abstração total.

\begin{Verbatim}
fun splitQuickSort: List a -> Split a a
fun joinQuickSort: Split b b -> List b
fun quickSort: List a -> List a
...
. splitQuickSort (Cons x xs) 
     = let (ys, zs) = partition (\t:a .! x < t) xs
       in Split x (Cons ys (Cons zs Nil))                    %(SplitQuickSort)%
. joinQuickSort (Split x (Cons ys (Cons zs Nil))) 
     = ys ++ (Cons x zs)                                      %(JoinQuickSort)%
. quickSort xs = genSort splitQuickSort joinQuickSort xs     %(QuickSort)%
\end{Verbatim}

O algoritmo \Verb.Selection Sort. faz uso de uma função de divisão que depende da função \Verb.minimum. para extrair o menor elemento de uma lista.

\begin{Verbatim}
fun splitSelectionSort: List a -> Split a a
fun joinSelectionSort: Split b b -> List b
fun selectionSort: List a -> List a
...
. splitSelectionSort xs = let x = minimum xs in
	Split x (Cons (delete x xs) (Nil: List(List a)))    %(SplitSelectionSort)%
. joinSelectionSort (Split x (Cons xs Nil)) =
 	(Cons x xs)                                           %(JoinSelectionSort)%
. selectionSort xs
     = genSort splitSelectionSort joinSelectionSort xs   %(SelectionSort)%
\end{Verbatim}

O algoritmo \Verb.Merge Sort. divide uma lista ao meio e, então, une as duas listas, ordenando-as recursivamente.

\begin{Verbatim}
fun splitMergeSort: List b -> Split b Unit
fun joinMergeSort: Split a Unit -> List a
fun merge: List a -> List a -> List a
fun mergeSort: List a -> List a
...
. def((length xs) div 2) /\ n = ((length xs) div 2) 
     => splitMergeSort xs = let (ys,zs) = splitAt n xs
        in Split () (Cons ys (Cons zs Nil))                  %(SplitMergeSort)%
. xs = (Nil: List a) => merge xs ys = ys                           %(MergeNil)%
. xs = (Cons v vs) /\ ys = (Nil: List a) 
     => merge xs ys = xs                                       %(MergeConsNil)%
. xs = (Cons v vs) /\ ys = (Cons w ws) /\ (v < w) = True 
     => merge xs ys = Cons v (merge vs ys)                   %(MergeConsConsT)%
. xs = (Cons v vs) /\ ys = (Cons w ws) /\ (v < w) = False 
     => merge xs ys = Cons w (merge xs ws)                   %(MergeConsConsF)%
. joinMergeSort (Split () (Cons ys (Cons zs Nil))) 
     = merge ys zs                                            %(JoinMergeSort)%
. mergeSort xs = genSort splitMergeSort joinMergeSort xs    %(MergeSort)%
\end{Verbatim}

A fim de verificar propriedades acerca das definições das funções, foram definidos três predicados.
O primeiro, \Verb.__elem__., verifica se um elemento pertence a uma lista;
o predicado \Verb.isOrdereded. garante que uma lista está ordenada de forma correta;
por fim, o predicado \Verb.permutation. verifica se uma lista é uma permutação de uma segunda lista, ou seja, verifica se ambas as listas possuem os mesmos elementos.
Embora o predicado \Verb.%(PermutationCons)%. fosse desnecessário para definir o predicado \Verb.permutation., o primeiro se mostrou necessário nas provas envolvendo este último.

\begin{Verbatim}
vars a: Ord;
     x,y: a;
     xs,ys: List a
preds __elem__ : a * List a;
      isOrdered: List a;
      permutation: List a * List a
. not x elem (Nil: List a)                                          %(ElemNil)%
. x elem (Cons y ys) <=> x = y \/ x elem ys                        %(ElemCons)%
. isOrdered (Nil: List a)                                      %(IsOrderedNil)%
. isOrdered (Cons x (Nil: List a))                            %(IsOrderedCons)%
. isOrdered (Cons x (Cons y ys))                    
     <=> (x <= y) = True /\ isOrdered(Cons y ys)       %(IsOrderedConsCons)%
. permutation ((Nil: List a), Nil)                           %(PermutationNil)%
. permutation (Cons x (Nil: List a),                
 	Cons y (Nil: List a)) <=> x=y                           %(PermutationCons)%
. permutation (Cons x xs, Cons y ys) <=>            
     (x=y /\ permutation (xs, ys)) \/ (x elem ys    
          /\ permutation(xs, Cons y (delete x ys)))   %(PermutationConsCons)%
\end{Verbatim}

Foram criados teoremas para garantir que a aplicação dos algoritmos sobre uma mesma lista, dois a dois, obtivessem a mesma lista como resultado;
para verificar que a aplicação de cada algoritmo em uma lista resultasse em uma lista ordenada;
e para provar que uma lista de entrada seja permutação da lista resultante da aplicação de cada função de ordenação sobre a lista inicial.

\begin{Verbatim}
then %implies                                       
var a,b : Ord;                                      
    xs, ys : List a;                                
. insertionSort xs = quickSort xs                                 %(Theorem01)%
. insertionSort xs = mergeSort xs                                 %(Theorem02)%
. insertionSort xs = selectionSort xs                             %(Theorem03)%
. quickSort xs = mergeSort xs                                     %(Theorem04)%
. quickSort xs = selectionSort xs                                 %(Theorem05)%
. mergeSort xs = selectionSort xs                                 %(Theorem06)%
. isOrdered(insertionSort xs)                                     %(Theorem07)%
. isOrdered(quickSort xs)                                         %(Theorem08)%
. isOrdered(mergeSort xs)                                         %(Theorem09)%
. isOrdered(selectionSort xs)                                     %(Theorem10)%
. permutation(xs, insertionSort xs)                               %(Theorem11)%
. permutation(xs, quickSort xs)                                   %(Theorem12)%
. permutation(xs, mergeSort xs)                                   %(Theorem13)%
. permutation(xs, selectionSort xs)                               %(Theorem14)%
end
\end{Verbatim}
