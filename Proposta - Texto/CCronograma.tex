\section{Cronograma}
A seguir, apresentam-se as atividades a serem realizadas durante o projeto de mestrado. A Tabela~\ref{tab:cronograma} possui o tempo estimado das atividades:

\begin{enumerate}
\item Cr�ditos obrigat�rios do mestrado; \label{cron:1}
\item Estudo das linguagens HasCASL e CASL, com �nfase nas constru��es de HasCASL e na biblioteca de CASL; \label{cron:2}
\item Estudo da ferramenta Hets e do provador Isabelle; \label{cron:3}
\item Exame de Qualifica��o do Mestrado; \label{cron:4}
\item Estudo da biblioteca Prelude e dos conceitos de l�gica de segunda ordem envolvidos na bilbioteca; \label{cron:5}
\item Implementa��o e prova de corretude da biblioteca para HasCASL. \label{cron:6}
\item Escrita de um artigo para congresso. \label{cron:7}
\item Escrita da disserta��o. \label{cron:8}
\item Defesa e revis�o. \label{cron:9}
\end{enumerate}

{
\small
\begin{table}[hb]
\begin{center}
\begin{tabular}{||c||c|c|c|c|c||c|c|c|c|c|c||c||}

\hline \hline
%% Anos
Atividade
& \multicolumn{5}{c||}{2007} 
& \multicolumn{6}{c||}{2008}
& \multicolumn{1}{c||}{2009} \\ \cline{2-13}

%% Meses
&  3-4 & 5-6 & 7-8 & 9-10 & 11-12 & 1-2 & 3-4 & 5-6 & 7-8 & 9-10 & 11-12 & 1-2 \\ \hline \hline

%% Atividades
\ref{cron:1}  & $\bullet$ & $\bullet$ & $\bullet$ & $\bullet$ & $\bullet$ & & & & & & & \\ \hline 
\ref{cron:2}  & & $\bullet$ & $\bullet$ & & & & & & & & & \\ \hline 
\ref{cron:3}  & & & & $\bullet$ & $\bullet$ & & & & & & & \\ \hline 
\ref{cron:4}  & & & & $\bullet$ & & & & & & & & \\ \hline 
\ref{cron:5}  & & & & & & $\bullet$ & $\bullet$ & & & & & \\ \hline 
\ref{cron:6}  & & & & & & & & $\bullet$ & $\bullet$ & $\bullet$ & & \\ \hline 
\ref{cron:7}  & & & & & & & & & & $\bullet$ & & \\ \hline 
\ref{cron:8}  & & & $\bullet$ & & $\bullet$ & & $\bullet$ & & & & $\bullet$ & \\ \hline 
\ref{cron:9}  & & & & & & & & & & & & $\bullet$ \\ \hline

\end{tabular}
\caption{Cronograma de Atividades: mar�o/2007 at� fevereiro/2009} 
\label{tab:cronograma}
\end{center}
\end{table}
}
