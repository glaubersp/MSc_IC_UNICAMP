\begin{abstract}
CASL � uma linguagem de especifica��o alg�brica criada para ser um padr�o. Atrav�s de exten��es e restri��es a CASL, cria-se uma fam�lia de linguagens que formam um arcabou�o de especifica��o alg�brica. A exten��o que possui elementos de l�gica de segunda ordem , chamada HasCASL, permite a especifica��o de programas funcionais com elementos muito parecidos com os presentes na linguagem de programa��o Haskell. Embora CASL j� possua uma biblioteca de especifica��es prontas, HasCASL ainda carece de tal biblioteca. Dado que uma linguagem s� � vastamente utilizada quando permite o reuso de c�dido, a presente proposta visa � cria��o de uma biblioteca para HasCASL baseada na biblioteca Prelude, da linguagem funcional Haskell. Prelude ser� tomada como base a fim de facilitar a gera��o de c�digo execut�vel a partir de uma especifica��o em HasCASL, j� que os elementos utilizados encontrar�o correspondentes na biblioteca Prelude.
\end{abstract}